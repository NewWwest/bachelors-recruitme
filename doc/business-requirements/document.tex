\documentclass{article}
\usepackage[T1]{fontenc}

\title{Wielowarstwowy system rekrutacji dla szkół z interfejsem webowym i aplikacją mobilną - analiza wymagań}
\author{Andrzej Westfalewicz, Filip Zyskowski}
\date{24 października 2019}

% zamienia nazwę Content na Spis treści
\renewcommand*\contentsname{Spis treści} 

\begin{document}

\begin{titlepage}
\maketitle
\end{titlepage}

\tableofcontents

\section{Opis systemu}

Celem projektu jest (zaliczenie inżynierki xd) utworzenie systemu służącego do rekrutacji uczniów do szkół średnich. Tworzony system ma za zadanie usprawnić proces przyjmowania kandydatów zarówno dla kadry nauczycielskiej jak i samych uczniów.

Głównymi funkcjami programu będzie rejestracja kandydata do systemu, przyjęcie opłaty rekrutacyjnej oraz zapisanie kandydata na wybrany przez siebie egzamin.

Dodatkowa funkcjonalność systemu będzie obejmowała dodanie zdjęcia kandydata, nadanie numeru egzaminacyjnego po poprawnym zapisaniu się na egzamin oraz pokazaniu stanu rekrutacji (przyjęty/odrzucony/w trakcie rozpatrywania)

Użytkownicy będą mieli dostęp do systemu zarówno z przeglądarki internetowej jak i z aplikacji mobilnej na telefonach z systemem Android.

\section{Słownik pojęć}

\begin{itemize}
	\item \textbf{System} - rozwiązanie mające na celu rekrutację uczniów do szkół średnich
	\item \textbf{Architektura} - podstawowa organizacja systemu wraz z jego komponentami, powiązaniami i regułami ustanawiającymi sposób budowy i rozwoju systemu
	\item \textbf{Szkoła} - instytucja oświatowo-wychowawcza będąca odbiorcą systemu
	\item \textbf{Użytkownik} - osoba korzystająca z systemu
	\item \textbf{Kandydat} - użytkownik, który chce wziąć udział w rekrutacji do szkoły
	\item \textbf{Aplikacja webowa (internetowa)} - program komputerowy, pracujący na serwerze i komunikujący się poprzez sieć komputerową z urządzeniem użytkownika z wykorzystaniem przeglądarki internetowej użytkownika
	\item \textbf{Aplikacja mobilna} - program działający na urządzeniach mobilnych (telefony komórkowe, smartfony, tablety)
	\item \textbf{Aplikacja} - aplikacja webowa lub aplikacja mobilna, która daje dostęp do systemu
	\item \textbf{Rejestracja} - proces przekazania danych kandydata do systemu poprzez aplikację oraz wpłacenia kwoty egzaminacyjnej
	\item \textbf{Egzamin} - forma sprawdzenia wiedzy kandydatów po udanym procesie rejestracji, będąca podstawą do przyjęcia bądź odrzucenia kandydata do szkoły
	\item \textbf{Kwota egzaminacyjna} - ustalona przez szkołę na okres rekrutacji kwota pieniężna, która musi być uiszczona przed ukończeniem procesu rejestracji
	\item \textbf{Okres rekrutacyjny} - okres ustalony przez szkołę, w którym można się rejestrować
	\item \textbf{ID kandydata} - unikalny identyfikator kandydata składający się z: 3 pierwszych liter imienia, 3 pierwszych liter nazwisk oraz 3 cyfr gwarantujących unikalność.
	\item \textbf{Organizator} - TODO.
\end{itemize}

\section{Wymagania funkcjonalne}

\subsection{Użytkownik}
\begin{enumerate}
  \item Jako Użytkownik mogę otworzyć stronę i przeglądać ogólnodostępną zawartość, aby zapoznać się ze szkołą i procesem rejestracji.   
      \begin{itemize}
         \item Aplikacja jest dostępna z internetu.
         \item Aplikacja nie rzuca błędów na konsolę.
         \item Aplikacja ładuje się "w rozsądnym czasie".
         \item Dostępne są takie strony jak: Regulamin rejestracji, kontakt do Organizatorów.
       \end{itemize}
\end{enumerate}

\subsection{Kandydat}
Funkcjonalności Kandydata działają również na aplikacji mobilnej.
\begin{enumerate}
    \item Jako Kandydat mogę się zarejestrować, aby wziąć udział w procesie rekrutacji  
        \begin{itemize}
        \item Podaje: Imię, nazwisko, email, hasło, numer PESEL (jeśli ma),
        \item Mail jest potwierdzony za pomocą przesłanego na niego linku.
        \item Dany email może być użyty maksymalnie 10 razy do utworzenia konta.
        \item Po potwierdzeniu adresu email Użytkownik dostaje swoje ID i utrzymuje dostęp do rejestracji.
        \end{itemize}
    \item Jako Kandydat mogę się zalogować, aby sprawdzić spersonalizowane informacje dotyczące rekrutacji.
        \begin{itemize}
        \item Logowanie odbywa się za pomocą ID kandydata oraz hasła.
        \item Aplikacja rozpoznaje czy obecny Użytkownik jest zalogowany.
        \item Część funkcjonalności jest dostępna tylko dla zalogowanych użytkowników.
        \end{itemize}
    \item Jako Kandydat mogę zresetować hasło, aby nie utrać dostępu do Systemu.    
        \begin{itemize}
        \item Za pomocą maila przesyłanego na wskazany  podczas rejestracji email.
        \end{itemize}
    \item Jako Kandydat przypomnieć ID Kandydata, aby nie utrać dostępu do Systemu. 
        \begin{itemize}
        \item Poprzez podanie wszystkich informacji podanych podczas rejestracji.
        \end{itemize}
    \item Jako zalogowany Kandydat otworzyć stronę swojego profilu aby edytować swoje dane.   
        \begin{itemize}
        \item Strona wyświetla dane podane przy rejestracji, jednak są one zablokowane do edycji.
        \item Wyświetlane są pozostałe dane do podania/edycji: imiona i nazwiska rodziców, nazwa szkoły podstawowej, adres zamieszkania.
        \item Zdjęcie.
        \item Dodatkowe dokumenty.
        \end{itemize}
    \item Jako zalogowany Kandydat mogę dokonać opłaty za rejestrację, aby dokończyć rejestrację.
        \begin{itemize}
        \item Strona profilu wyświetla obecny stan transakcji.
        \item Transakcja odbywa się przez internetową bramkę płatniczą.
        \item Po potwierdzeniu płatności Kandytat jest przypisany do Egzaminów.
        \end{itemize}
    \item Jako Kandydat, który dokonał płatności, mogę sprawdzić terminy swoich egzaminów, aby na nie pójść. XD
        \begin{itemize}
        \item Wyświetla się termin, miejsce, rodzaj egzaminu oraz typ.
        \end{itemize}
    \item Jako Kandydat, mogę zadać pytanie organizatorom poprzez stronę internetową, aby ułatwić komunikację ze Szkołą.
        \begin{itemize}
        \item Zadane pytania wyświetlają się w formie chatu. (Guest communication v2 :D)
        \item Kiedy jest otwarta dana zakładka, strona wiadomości synchronizują się co 15 sekund.
        \end{itemize}
    \item Jako Kandydat, mogę sprawdzić stan rekrutacji - przyjęty, nieprzyjęty, w trakcie.
        \begin{itemize}
        \item TODO
        \end{itemize}
\end{enumerate}

\subsection{Organizator Rejestracji}
\begin{enumerate}
  \item Zalogować się do Aplikacji i rozpoznać że jest organizatorem
      \begin{itemize}
         \item TODO
       \end{itemize}
  \item Pełen CRUD na Bazie
      \begin{itemize}
         \item TODO
       \end{itemize}
  \item Chat adminowy
      \begin{itemize}
         \item TODO
       \end{itemize}
  \item Zarządzanie Egzaminami
      \begin{itemize}
         \item Na egzamin składa się: rodzaj Egzaminu, Forma Egzaminu, Miejsce, data i godzina rozpoczęcia, czas trwania, limit kandydatów.
         \item Po dokonaniu opłaty Kandydatowi jest przypisywany po jednym Egzaminie z każdego rodzaju. Jeśli egzamin jest pisemny, godziną egzaminu jest godzina rozpoczęcia, jeśli egzamin jest ustny to jest to  godzina rozpoczęcia + czas trwania * numer na liście.
         \item Organizator może zmieniać Egzaminy Kandydata.
       \end{itemize}
\end{enumerate}

\subsection{Skanowanie kart odpowiedzi}
\begin{enumerate}
  \item TODO
      \begin{itemize}
         \item TODO
       \end{itemize}
\end{enumerate}

\section{Wymagania niefunkcjonalne}

\begin{itemize}
	\item todo: SECUTRITY.
	\item Aplikacja będzie dostępna przez co najmniej dwadzieścia dwie godziny w ciągu doby w okresie rekrutacyjnym.
	\item Architektura zapewnia bezproblemowe równoległe korzystanie z systemu przez co najmniej stu użytkowników.
	\item Przy połączeniu szerokopasmowym, wywołany przez użytkownika interfejs otworzy się po czasie nie dłuższym niż trzy sekundy. 
	\item Aplikacja będzie działała bezproblemowo na przeglądarkach internetowych Google Chrome, Opera, Mozilla Firefox, Safari, Microsoft Edge oraz Internet Explorer od wersji odpowiednio 60, 50, 60, 10.1, 15 oraz 11. Aplikacja będzie działała bezproblemowo na telefonach komórkowych z systemem Android od wersji 4.1.
	\item Aplikacja mobilna nie zadziała na telefonach, w których nie ma możliwości zainstalowania zewnętrznych aplikacji.
	\item Aplikacja webowa i mobilna do poprawnego działania wymaga połączenia z Internetem. Na urządzeniach bez możliwości połączenia z Internetem, aplikacja webowa i mobilna nie zadziałają.
\end{itemize}

Aplikacja mobilna do swojego działania wykorzystuje możliwości przez obecne urządzenia mobilne. Aby zapewnić poprawne działanie aplikacji mobilnej, przy instalacji aplikacji mobilnej lub przy pierwszym użyciu funkcji aplikacji korzystającego z danego modułu telefonu, użytkownik będzie proszony o zgodę na:
\begin{itemize}
	\item \textbf{odbieranie danych z internetu i pełny dostęp do sieci} – na potrzeby prawidłowej rejestracji, dokonania płatności i zapisania na egzamin,
	\item \textbf{wyświetlanie połączeń sieciowych} – na potrzeby sprawdzania dostępu do internetu przez aplikację,
\end{itemize}

\section{Harmonogram}

Harmonogram przedstawia przewidywany plan działania dla projektu. Tabelka zawiera piętnaście, dwukolumnowych wierszy. W pierwszej kolumnie każdego wiersza poniższej tabelki znajdziemy numer tygodnia, który jest rozparywany. Druga kolumna natomiast zawiera listę zadań, które mają być wykonane w danym tygodniu.

\begin{center}
	\begin{tabular}{|c|c|}
	\hline
	Numer tygodnia & Lista zadań \\
	\hline
	Tydzień 1. & "" \\
	\hline
	Tydzień 2. & "" \\
	\hline
	Tydzień 3. & "" \\
	\hline
	Tydzień 4. & "" \\
	\hline
	Tydzień 5. & "" \\
	\hline
	Tydzień 6. & "" \\
	\hline
	Tydzień 7. & "" \\
	\hline
	Tydzień 8. & "" \\
	\hline
	Tydzień 9. & "" \\
	\hline
	Tydzień 10. & "" \\
	\hline
	Tydzień 11. & "" \\
	\hline
	Tydzień 12. & "" \\
	\hline
	Tydzień 13. & "" \\
	\hline
	Tydzień 14. & "" \\
	\hline
	Tydzień 15. & "" \\
	\hline
	\end{tabular}
\end{center}

\end{document}